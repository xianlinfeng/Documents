\documentclass[11pt]{article}
\usepackage[utf8]{inputenc}

\usepackage{amssymb}
\usepackage{amsmath}
\usepackage{amsthm}

\theoremstyle{definition}
\newtheorem{definition}{Definition}[section] % add the section number in front of the definition number
% \newtheorem*{definition}{Definition} % remove the index number of the definition

\title{Pseudo Code for Repair Algorithm}
\author{Arthur Feng}
\date{\today}

\begin{document}
\maketitle

\section{Input}
\begin{itemize}
    \item A problem $P^0$ with $n$ binary variables $x$, objective function $c x$ (min), constraint set $C^0$ with an optimal solution $x^0$.
    \item A problem $P^1$ with $n$ binary variables $x$, objective function $c x$ (min), constraint set $C^1$, such that $C^0 \subsetneq C^1$.
\end{itemize}							

\section{Output}
\begin{itemize}	
    \item An optimal solution $x^1$ to $P^1$.
\end{itemize}

\section{General Idea}
We propose to solve $P^1$ by reusing the optimal solution $x^0$ to $P^0$.
In order to achieve this, we define a new problem $Q$ with constraint set $C^1$ and objective function
\begin{align*}
    \min c x + \alpha \lvert x - x^0 \rvert,					
\end{align*}
where $\lvert x - x^0 \rvert = \sum_{i=0}^{n-1}\lvert x_i - x^0_i \rvert$, and $\alpha$ is a \emph{penalty} term for deviating from the input solution $x^0$.
This would tentatively help the search for a good solution to $P^1$.
However, unless $x^0$ is feasible for $P^1$, an optimal solution to $Q$ will in general not be optimal for $P^1$.

To remedy this problem, we will instead solve a sequence of problems $Q^0, Q^1, \dots$, where the penalty factor $\alpha$ will gradually decrease until it reaches 0, say at iteration $k$, in which case $Q^k = P^1$.
This sequence of problems can be efficiently solved using a technique called \emph{reoptimisation}, which is implemented in the MIP solver SCIP.

\section{Pseudo Code}


\section{MIP}
\subsection{Mixed Integer Programming}
\begin{definition}
(Mixed Integer Programming)\\
Let $m,n \in \mathbb{N}$, The given matrix $A\in \mathbb{R}^{m \times n}$, vectors $b\in \mathbb{R}^m$, and the vector $c \in \mathbb{R}^n$, and a set $\mathcal{I} \subseteq N = \{ 1,\dots,n \} $. the problem 
\begin{align*}
\text{(MIP)} \hspace{5mm} &c^* = \text{min}\,c^Tx& \\
&s.t.\hspace{5mm} Ax \geq b  \\
&\hspace{11mm} x_i \in \mathbb{Z}_{\geq0} \hspace{3mm} \forall_i \in \mathcal{I}\\ 
&\hspace{11mm} x_j \in \mathbb{R}_{\geq0} \hspace{3mm} \forall_i \in \mathcal{N}\,\backslash\,  \mathcal{I}
\end{align*}
is called $mixed integer program$ with the objective function $c^Tx$ and constraints $A_ix \geq b_i$ for all $i = 1,\dots ,m$.
A vector $x\in X_{MIP} = \{x \in \mathbb{R}_\geq0^n\,|\,A_x \geq b, \, x_i \in \mathbb{Z}_{\geq0}\,\forall i \in \mathcal{I}  \}$ is called $feasiable solution$ and $X_{MIP}$ the set of feasible solutions. A feasible $x^*$ is called optimal, if $x^*$ satisfies $c^*=c^Tx^*$.

\end{definition}




\end{document}
