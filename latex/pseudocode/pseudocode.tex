\documentclass[11pt]{article}
\usepackage[utf8]{inputenc}

\usepackage{amssymb}
\usepackage{amsmath}
\usepackage{amsthm}
\usepackage{fullpage}
\usepackage[ruled,vlined,linesnumbered]{algorithm2e}

\setlength{\parindent}{0em}  % set the indent of the paragraph
\setlength{\parskip}{1em} % set the space of the paragraph

\theoremstyle{definition}
\newtheorem{definition}{Definition}[section] % add the section number in front of the definition number
% \newtheorem*{definition}{Definition} % remove the index number of the definition

\title{Pseudo Code for Repair Algorithm}
\author{Arthur Feng}
\date{\today}

\begin{document}
\maketitle

\section{Introduction}
\begin{definition}
(Mixed Integer Programming)\\
In this report, we denote the $Mixed\,Integer\, programming$ (MIP) problem in the follow form $$\text{(MIP)}  \hspace{3mm} \text{min} \{\, c^Tx : Ax \leq b, x \in \mathbb{Z}^l\,\text{x}\,\mathbb{R}^k \}  $$ 
where $m,n \in \mathbb{N}$, and $n = l + k $,  $A$ is a matrix for $A \in \mathbb{R}^{m \,\text{x}\, n}$, and the vector $b \in \mathbb{R}^m$, the vector $c \in \mathbb{R}^n$. \par
$S$ is a set of $feasible\,solution$ if $S$ satisfy all the constraints in the problem. If $s \in S$, $s$ is called a feasible solution to the problem $\mathcal{Z}_{MIP}$. The MIP is called infeasible when $S$ is empty, other wise, the MIP is called feasible. The feasible solution $s^\ast$ is called optimal when $s^\ast \in S $ and  $ c^Tx_{s^\ast} \leq c^Tx_s \text{ for } \forall s \in S $. \par
% ref: https://eprint.iacr.org/2012/676.pdf
% ref: Mixed-integer Linear Programming in the Analysis of Trivium and Ktantan
\label{dif:def_MIP}
\end{definition}

\begin{definition}
(Linear Programming and LP-relaxation)\\
With the definition of MIP in \ref{dif:def_MIP}, there is a special case of MIP when all variables are continuous, which is called $Linear Programming $ (LP) problem
$$\mathcal{Z}_{LP} = \text{min} \{\, c^Tx : Ax \leq b, x \in \mathbb{R}^k \}  $$ 
A LP can also be obtained by removing all integrity constraints: $x_i\, \in\, x \text{ where } i \in n \setminus l $, this is called LP-$relaxation$. LP-$relaxation$ is the foundation of LP-based branch-and-bound technology. As the searching space is increase by removing integrity restrictions, the optimal solution in MIP could not better than LP-$relaxation$, which is  $s_{MIP}^\ast \geq s_{LP}^\ast $. This means the optimal solution found in LP could provide a lower or prime bound for MIP.
\label{dif:def_lp}
\end{definition}

\begin{definition}
(Binary Programs)\\
In the definition \ref{dif:def_MIP}, When all variables are integer
$$\mathcal{Z}_{IP} = \text{min} \lbrace\, c^Tx : Ax \leq b, x \in \mathbb{Z}^l \rbrace $$ 
The problem is call pure Integer Programming (IP) problem. And if we denote $\mathbb{B}$ be a set of binary values where $\mathbb{B} = \{0, 1 \} \text{ for } \forall x \in \mathbb{B}^l$, then we call this problem a Binary Programming problem.  This report we focus on Binary Programs.\\
Question: the two terminologies $Binary programming problem$ and $pseudo-Boolean problem$ is same? or not?
\label{dif:def_bp}
\end{definition}


\section{Input}
\begin{itemize}
    \item A problem $ \mathcal{P}^0 = min \{ c^Tx\,|\, Ax \geq b,\, x_i \in \mathbb{B}^n, \mathbb{B} = \lbrace 0,1\rbrace  \}$ with $n$ binary variables $x$, objective function $ O^0 = c^T x$ (min), constraint set $C^0$ with an optimal solution $s^0$.
    \item A problem $\mathcal{P}^1$ with $n$ binary variables $x$, objective function $O^1 = O^0 = c^Tx$ (min), constraint set $C^1$, such that $C^0 \subsetneq C^1$.
\end{itemize}							

\section{Output}
\begin{itemize}	
    \item An optimal solution $s^1$ to $\mathcal{P}^1$.
\end{itemize}

\section{Pseudo Code}



\begin{algorithm}[H]
\SetAlgoLined
\DontPrintSemicolon
\KwData{$r, \mathcal{P}^0 \text{,} \,  s^0 $ and  $ \mathcal{P}^1$ where $C^0 \subsetneq C^1$ , $O^0 = O^1$}
\KwResult{ optimal solution $s^1 \text{ to } \mathcal{P}^1 $}
\Begin{
	\eIf{$s^0 \text{ is feasible to} \mathcal{P}^1$}{
			\textbf{return} $s^0$\;
 		} {
 			$C^+ \longleftarrow C^1 - C^0$\;
 			\For{$c \in C^+ $}{
				add $v_i$ to the set $V$ where it's coefficient $c_i \neq 0$\;
				% get the set of changed variables 
 				}
 			
 			% different function to test solve the problem from scratch or not  	
 			$y^0 = c^Tx \text{ for } x \in s^0$\;
 			$y^p = c^Tx \text{ for } x \in s^0 \, \cap\, V^+$\;  % the biggest distance is when all x_i in V^+ are different from s^0
			$\Delta y =|Y^0 - Y^p | $ \;	
			\eIf{$\Delta y / y^0 \geq r $ % when y is greater than distance rate }{	
				solve $\mathcal{P}^1$ from scratch\;
				} {
				solve it with reoptimization\;	
				$p \longleftarrow ( | y^0 | + |y^p|) $ \;% just set a penalty start number
				$O^p \longleftarrow p * \sum x \text{ for } x \in V^+ $ \;
								
				
				
				
				$\mathcal{P}^* \longleftarrow O^p + C1 $
				
				
	 			% P^p <--- O^1 + O^p <--  generate the penalty function and coefficient 

	 			% try to repair s^0 to P^P to get the first feasible solution
	 				 			
	 			% reoptimize the problem and every iteration to decrease the coefficient of O^p
	 			% decrease reate: p = p // 1.2 
	 			$V^+ \longleftarrow for v in C^+ where c$
	 			}
	 			
	 			$V \longleftarrow U$\;
$S \longleftarrow \emptyset$\;
\For{$x\in X$}{
    $NbSuccInS(x) \longleftarrow 0$\;
    $NbPredInMin(x) \longleftarrow 0$\;
    $NbPredNotInMin(x) \longleftarrow |ImPred(x)|$\;
    }
\For{$x \in X$}{
    \If{$NbPredInMin(x) = 0$ {\bf and} $NbPredNotInMin(x) = 0$}{
        $AppendToMin(x)$}
    }
    \While{$S \neq \emptyset$}{\label{InRes1}
        remove $x$ from the list of $T$ of maximal index\;
    \While{$|S \cap ImSucc(x)| \neq |S|$}{
    \For{$ y \in S-ImSucc(x)$}{
        \{ remove from $V$ all the arcs $zy$ : \}\;
        \For{$z \in ImPred(y) \cap Min$}{
            remove the arc $zy$ from $V$\;
            $NbSuccInS(z) \longleftarrow NbSuccInS(z) - 1$\;
            move $z$ in $T$ to the list preceding its present list\;
            \{i.e. If $z \in T[k]$, move $z$ from $T[k]$ to
                $T[k-1]$\}\;
            }
            $NbPredInMin(y) \longleftarrow 0$\;
            $NbPredNotInMin(y) \longleftarrow 0$\;
            $S \longleftarrow S - \{y\}$\;
            $AppendToMin(y)$\;
            }
        }
        $RemoveFromMin(x)$\;
    }			
 		}
}
 \caption{Solving Problem with Reoptimization}
\end{algorithm}

\begin{algorithm}[H]
\SetAlgoLined
\KwResult{Write here the result }
 initialization\;
 \While{While condition}{
  instructions\;
  \eIf{condition}{
   instructions1\;
   instructions2\;
   }{
   instructions3\;
  }
 }
 \caption{How to write algorithms}
\end{algorithm}





\section{MIP}
\subsection{Mixed Integer Programming}
\begin{definition}
(Mixed Integer Programming)\\
Let $m,n \in \mathbb{N}$, The given matrix $A\in \mathbb{R}^{m \times n}$, vectors $b\in \mathbb{R}^m$, and the vector $c \in \mathbb{R}^n$, and a set $\mathcal{I} \subseteq N = \{ 1,\dots,n \} $. the problem 
\begin{align*}
\text{(MIP)} \hspace{5mm} &c^* = \text{min}\,c^Tx& \\
&s.t.\hspace{5mm} Ax \geq b  \\
&\hspace{11mm} x_i \in \mathbb{Z}_{\geq0} \hspace{3mm} \forall_i \in \mathcal{I}\\ 
&\hspace{11mm} x_j \in \mathbb{R}_{\geq0} \hspace{3mm} \forall_i \in \mathcal{N}\,\backslash\,  \mathcal{I}
\end{align*}
is called $mixed integer program$ with the objective function $c^Tx$ and constraints $A_ix \geq b_i$ for all $i = 1,\dots ,m$.\par
\label{dif:mip}
\end{definition}
A vector $x\in X_{MIP} = \{x \in \mathbb{R}_\geq0^n\,|\,A_x \geq b, \, x_i \in \mathbb{Z}_{\geq0}\,\forall i \in \mathcal{I}  \}$ is called $feasiable solution$ and $X_{MIP}$ the set of feasible solutions. A feasible $x^*$ is called optimal, if $x^*$ satisfies $c^*=c^Tx^*$. \par
Common special cases of MIPs are $linear programs$ (LPs) for $\mathcal{I} = \emptyset $ and $integer program$ (IPs) for $\mathcal{I} = N$. Additional, an integer variable bounded by 0 and 1 is called $binary variable$. let $\mathcal{B} \subset \mathcal{I}$ denote the set of binary variables. An integer program with $\mathcal{B} = N $ is called $binary program$ (BP) or $mixed binary program$ (MBP), if $\mathcal{B} = \mathcal{I} \subsetneq N$. \par
A lower or dual bound on a MIP can be computed by neglecting the intergrality constraints. the so-obtained problem is called the $LP\text{-}relaxation$ of the MIP. \par

\begin{definition}
(LP-relaxation)\\
Given a MIP as introduced in Definition \ref{dif:MIP_def}. The LP-$relaxation$ is defined as
	\begin{align*}
	\text{(MIP)} \hspace{5mm} &c^* = \text{min}\,c^Tx& \\
	&s.t.\hspace{5mm} Ax \geq b  \\
	&\hspace{11mm} x \in \mathbb{R}_{\geq0}^n 
	\label{dif:LP}
	\end{align*}
Analogous to $X_{MIP}$ we can define $X_{LP}$ as the set of feasible solutions of the LP-relaxation. A feasible solution $x_{LP}^* \in X_{LP}$ is called LP-$optimal$ if $c_{LP}^* = c^Tx_{LP}^*$. In general solving MIPs is $NP$-hard. One common method for solving MIPs is LP-$based branch-and-bound$. This method splits the problem into smaller subproblems, and procedure is repeated on these subproblems. At any point a global upper or primal bound is given by the best known solution, if existent, and a local lower bound or dual bound is given by the respective LP-relaxation.\par

\subsection{Pseudo-Boolean optimization}
 In the section 2.4 we will present a special kind of a binary problem, a so-called $pseudo-Boolean problem$. For this purpose we introduce the basic definition of a $pseudo-Boolean problem$ in this section.For more detail we refer to Hammer and Rubin and Boros and Hammer and the references therein.\par
 Let us denote by $\mathbb{B} = \{0,1 \}$ the set of binary values and let $N = \{ 1, \dots ,n  \}$ be an index set, Reflecting to Boros and Hammer we consider functions in $n$ binary variables $x_1, x_2\dots ,x_n$ and denote the binary vector by $(x_1, x_2, \dots,x_n in \mathbb{B}^n)$. A function $f\,:\, \mathbb{B}^n \rightarrow\mathbb{R}$ of the form 
 $$\displaystyle{f(x_1,x_2,\dots,x_n) = \sum \limits_{S \subseteq N} C_S  \prod \limits_{i \in S}x_i}$$
 where $C_s \in \mathbb{R}$ for each $ S \subseteq N$, is called $pseudo-Boolean function$. see e.g., Hannmer et al. The degree of the function $deg(f)$ is given by the seize of the largest set $S \subseteq	N$ which$C_S \neq 0$.A pseudo-Boolean function $f$ is called linear, quadratic, cubic etc. if $deg(f) \leq 1,2,3 $ etc. repectively. Liu and Truszczynski defined a $pseudo-Boolean constraint $ as an integer inequality of the form 
 $$\sum \limits_{i=1}^{n} a_ix_i \geq b$$
 with $a_i,b in \mathbb{Z}$ and $xi\in \mathbb{B}$ for all $i in [n]$. A binary problem defined by a pseudo-Boolean objective function and a set of pseudo-Boolean constraints is called $pseudo-Boolean\, optimization\, problem$. \par


	
	 
 
\end{definition}


\end{document}
